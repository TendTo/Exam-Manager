\section{Conclusione}

Quando si determina l'implementazione di una soluzione software ad un problema,
spesso le possibilità sono numerose, ciascuna con vantaggi e difetti diversi. \\
L'approccio che abbiamo scelto ci ha permesso di ottenere una grande flessibilità
in termini di configurazione dei test e delle relazioni che hanno gli uni con gli altri.
Tuttavia è innegabile che, al fine di fornire le stesse funzionalità per il frontend,
sarebbe stato sufficiente eliminare tutta la parte relativa al controllo e la verifica dei dati,
che verrebbe relegata unicamente al professore, e limitare il ruolo del contratto al logging e allo storage dei dati storici immutabili. \\
La necessità di dipendere da un admin fidato sembra essere inevitabile,
in quanto è necessario avere accesso ad informazioni non disponibili unicamente all'interno della blockchain,
ma si tratta di una concessione limitata e comunque in misura minore a quella affidata in un sistema centralizzato. \\
Grazie all'utilizzo della blockchain, tutti gli eventi importanti sono consultabili in qualsiasi momento da chiunque,
garantendo accountability e traceability. \\
La \gls{dapp} migliora la usability dell'applicazione.
Sarebbe comunque possibile utilizzare lo direttamente \gls{smart-contract} da servizi come il block explorer della rete utilizzata.