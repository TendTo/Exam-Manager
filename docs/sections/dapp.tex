\section{Decentralized App: ExamManager}

\subsection{Cos'è una DApp e a cosa serve}

Per poter utilizzare le funzionalità di uno \gls{smart-contract} è necessario che un \gls{eoa} produca una transazione che ne invochi un metodo. \\
In linea di principio, chiunque potrebbe avviare un proprio nodo che si interfacci con la blockchain ed utilizzare quello per interagire con lo \gls{smart-contract}.
Tuttavia i requisiti, sia in termini di hardware che di conoscenze, sono proibitivi per la maggior parte degli utenti. \\
Al fine di rendere quanto più semplici ed intuitive questo genere di operazioni, spesso ad uno \gls{smart-contract} viene associata una \gls{dapp}. \\
Generalmente si tratta di applicazioni web che utilizzano le \gls{rpc-api},
approccio che si sposa perfettamente con un client che svolge il suo compito confinato nel browser dell'utente.
L'unico software che è necessario affiancare ad un qualsiasi browser è un wallet manager come \gls{metamask}, che è possibile installare come estensione. \\
Il wallet manager consente di gestire i vari account impedendo inoltre al sito web visitato di accedere direttamente a dati particolarmente sensibili quali le chiavi private.
Agli script della pagina web viene offerta un'API che permette loro di ottenere una transazione firmata, a patto che l'utente abbia dato il suo consenso esplicito per ognuna di esse.

\subsection{Implementazione}

L'implementazione presentata utilizza il framework NextJs \cite{soft:nextjs} e le librerie React \cite{soft:react}, DaisyUI \cite{soft:daisyui} ed ethers \cite{soft:ethers}. \\
Il risultato è un'applicazione web consultabile da browser che permette agli utenti di connettersi ed interagire con lo \gls{smart-contract} ExamContract.

\subsubsection{NextJs, React e DaisyUI}

NextJs è un framework JavaScript per applicazioni React.
Quest'ultima è la libreria più utilizzata fra tutti i tool del suo genere pensati per lo sviluppo frontend \cite{art:react-first}.
Grazie al suo sistema di hook per la gestione dello stato, negli anni si è dimostrato un'ottimo strumento per realizzare single page applications. \\
Sebbene il suo punto di forza sia il \gls{ssr}, NextJs permette anche l'esportazione statica dell'intero progetto.
L'output è una collezione di file HTML, CSS e JavaScript che possono essere serviti da un web server statico. \\
Nel nostro caso abbiamo utilizzato quello messo a disposizione da GitHub. \\
DaisyUI è una libreria css che utilizza TailWind \cite{soft:tailwind}.
È stata utilizzata per migliorare la user experience ed utilizzare un tema grafico coerente.

\subsubsection{EtherJs}

Etherjs è una libreria in grado semplificare e rendere più agevole l'utilizzo delle \gls{rpc-api} fornite dai nodi della blockchain.
Fornisce svariate astrazioni che nascondo i dettagli implementativi dietro a chiamate di metodi di un oggetto proxy. \\

\subsection{Utilizzo}

La \gls{dapp} riconosce l'utente che la utilizza in base all'indirizzo pubblico che ha utilizzato per connettervisi.
Consultato lo \gls{smart-contract}, stabilisce poi il ruolo dell'utente che può essere uno fra i seguenti:

\begin{itemize}
    \item Studente
    \item Professore
    \item Admin
    \item Sconosciuto
\end{itemize}

L'interfaccia e le operazioni consentite sono diverse per ognuno di questi e riflettono fedelmente la divisione di ruoli stabilita nello \gls{smart-contract} (\autoref{sec:roles}).

\subsubsection{Studente}

Allo studente vengono mostrate due liste distinte.

\begin{itemize}
    \item La prima contiene i corsi per i quali ha superato almeno una prove o per i quali ha ricevuto una proposta di voto.
    \item La seconda mostra tutti i corsi verbalizzati che ha superato con successo.
\end{itemize}

Lo studente ha la possibilità di rifiutare un voto di una prova che aveva precedentemente superato,
con la condizione che non abbia preso parte in una prova successiva che aveva come requisito il superamento della prima. \\
Nel momento in cui riceve una proposta di voto per la materia, lo studente può decidere se accettare o rifiutarla.
Nel caso in cui la proposta di voto sia accettata, la materia viene verbalizzata e non può più essere modificata, prove comprese.
Se invece la proposta di voto viene rifiutata, la proposta di voto viene invalidata, così come i risultati di tutte le prove che la compongono.

\subsubsection{Professore}

Il professore visualizza una lista di tutti i corsi che insegna attualmente e che è quindi autorizzato a gestire.
Tramite le apposite sezioni è in grado di modificare le prove previste in un corso, specificando per ciascuna di esse

\begin{itemize}
    \item Il nome della prova
    \item Il voto minimo necessario per il superamento la prova
    \item Il tempo di validità del risultato della prova
    \item La lista degli id dei test che lo studente deve aver superato per accedere alla prova
    \item La lista degli id dei test che verranno invalidati nel momento in cui lo studente ha partecipato alla prova
    \item La lista degli id dei test che verranno invalidati nel momento in cui lo studente fallisce o rifiuta la prova
\end{itemize}

Dopo una prova o a conclusione di un appello, il professore può preparare la lista dei risultati di ogni studente, identificati con una coppia matricola-voto,
e poi inviare il tutto allo \gls{smart-contract} per la registrazione in una sola transazione.

\subsubsection{Admin}

L'admin ha accesso a tutte le funzioni amministrative dello \gls{smart-contract}.

\begin{itemize}
    \item Aggiungere un nuovo studente
    \item Rimuovere uno studente
    \item Aggiungere una nuova materia
    \item Assegnare un professore ad una materia
    \item Rimuovere il professore precedentemente assegnato ad una materia
\end{itemize}
