\section{Introduzione}
Durante il percorso universitario, si è notato come il sistema di valutazione sia fortemente centralizzato, con la necessità di fiducia nell'intermediario, l'ateneo.
In fase di progettazione, inoltre, è stata rilevata una componente di fiducia necessaria nei confronti del prof per quanto riguarda i voti intermedi di una materia.
Prendendo in esempio i voti delle prove in itinere o di laboratorio, a differenza dei voti finali, tipicamente i primi non vengono verbalizzati.\\
\\
L'obiettivo di questo progetto è la decentralizzazione dei risultati degli esami universitari, attraverso l'utilizzo di uno \gls{smart-contract} che non necessiti di terze parti.\\
\\
Inoltre, per rendere più completo il contratto, si è deciso di renderlo più `intelligente', sfruttandolo non solo come strumento di logging, ma aggiungendo una componente di controlli, come ad esempio l'impossibilità di verbalizzare materie e prove con propedeuticità non rispettate. Inoltre è stata resa dinamica la fase di verbalizzazione del voto finale, creando un sistema di proposta-accettazione tra professore e studente. \\
\\
In fine, per rendere più user-friendly l'utilizzo del contratto, è stata creata un'app decentralizzata, che attraverso l'utilizzo di un wallet come \gls{metamask} permettesse di utilizzare tutte le funzioni dello smart contract, con delle pagine web dedicate e diverse in base al ruolo (Pubbliche, Studente, Prof, Admin). Nelle varie sezioni della pagina, che verranno descritte in seguito, si potranno trovare form o bottoni in grado di generare le relative transazioni automaticamente.\\
\\
Inoltre, per garante un'ottima usabilità, sono state utilizzate tecnologie responsive, ovvero in grado di adattarsi automaticamente al tipo di device, come PC, tablet e smartphone.\\