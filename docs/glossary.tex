\newglossaryentry{ethereum}
{
    name={Ethereum},
    text={Ethereum},
    description={
            Ethereum è una piattaforma decentralizzata in grado di eseguire istruzioni sotto forma
            di \gls{smart-contract}.
            Per questo viene spesso paragonata ad una macchina virtuale distribuita. \\
            Lo stato della macchina istante per istante è costruito a partire da quello
            precedente seguendo il protocollo di consenso.
            Ciò assicura che tutti gli utenti che seguono il protocollo ottengano lo stesso
            risultato.
        }
}
\newglossaryentry{smart-contract}
{
    name={Smart contract},
    text={smart contract},
    plural={smart contracts},
    description={
            Uno smart contract è un programma in grado di essere eseguito sulla blockchain \gls{ethereum}.
            Come ogni altra cosa in ambito blockchain, una volta entrato a far parte dei quest'ultima
            il codice non può più essere modificato. \\
            Tuttavia gli smart contract hanno la possibilità di possedere uno stato interno che può essere
            alterato con l'invocazione di uno dei suoi metodi da parte di un \gls{eoa}.
        }
}
\newglossaryentry{eoa}
{
    name={Externally Owned Account (EOA)},
    first={Externally Owned Account (EOA)},
    text={EOA},
    plural={EOAs},
    description={
            Sono account della blockchain \gls{ethereum} identificati da un indirizzo pubblico associato ad una chiave privata.
            Chiunque sia in possesso della chiave privata è in grado di firmare transazioni a nome dell'account. \\
            Per invocare un metodo di uno \gls{smart-contract}, è necessario che la transazione sia stata originata da un account EOA.
        }
}
\newglossaryentry{dapp}
{
    name={Decentralized Application (DApp)},
    first={Decentralized Application (DApp)},
    text={DApp},
    plural={DApps},
    description={
            Le DApp si differenziano dalle applicazioni tradizionali in quanto scelgono di evitare il modello client-server
            per appoggiarsi su piattaforme blockchain e il loro network distribuito. \\
            Nella maggior parte dei casi si tratta di applicazioni web.
        }
}
\newglossaryentry{rpc-api}
{
    name={JSON-RPC API},
    text={JSON-RPC API},
    description={Per poter interagire con la blockchain, è necessario passare attraverso un nodo che ne faccia parte.
            Poiché possederne e mantenerne uno potrebbe essere uno sforzo proibitivo per l'utente medio, esistono dei nodi pubblici
            che mettono a disposizione un'interfaccia JSON-RPC, trasmettendo poi le richieste alla rete di nodi. \\
            Questo permette ai client di invocare metodi di uno \gls{smart-contract} o di conoscere lo stato della blockchain tramite una semplice chiamata HTTP, HTTPs o websocket.
        }
}
\newglossaryentry{ssr}
{
    name={Server-side rendering (SSR)},
    first={Server-side rendering (SSR)},
    text={SSR},
    description={
            Contrariamente ad un sisto statico, che si limita ad inviare al client un file HTML preesistente in risposta alla sua richiesta,
            il server side rendering, come dice il nome, prevede che la pagina HTML sia generata nel momento della richiesta, avendo quindi la possibilità di integrare informazioni esterne come i dati di un database. \\
        }
}
\newglossaryentry{metamask}
{
    name={MetaMask},
    description={MetaMask è un'estensione o un plugin per browser web che permette agli utenti di interagire facilmente con le dApp del blockchain di Ethereum.
            Questo è possibile, perché MetaMask funge da ponte tra dApp e browser web, facilitandone l'utilizzo e il divertimento.}
}
\newglossaryentry{event-log}
{
    name={Event Log},
    text={event log},
    description={Output che uno \gls{smart-contract} può produrre con la codeword \texttt{emit}.
            L'output prodotto viene aggiunto in maniera permanente alla blockchain e può essere consultato da chiunque. \\
            È possibile inserire un numero arbitrario di parametri che saranno visibili nel log, e fino a tre parametri indicizzati,
            che possono essere usati per filtrare solo i log pertinenti. \\
            Possono rappresentare un'alternativa più economica per lo storage dei dati.
        }
}
