\newglossaryentry{ethereum}
{
    name={Ethereum},
    text={Ethereum},
    description={
            Ethereum è una piattaforma decentralizzata in grado di eseguire istruzioni sotto forma
            di \gls{smart-contract}.
            Per questo viene spesso paragonata ad una macchina virtuale distribuita. \\
            Lo stato della macchina istante per istante è costruito a partire da quello
            precedente seguendo il protocollo di consenso.
            Ciò assicura che tutti gli utenti che seguono il protocollo ottengano lo stesso
            risultato.
        }
}
\newglossaryentry{smart-contract}
{
    name={Smart contract},
    text={smart contract},
    plural={smart contracts},
    description={
            Uno smart contract è un programma in grado di essere eseguito sulla blockchain \gls{ethereum}.
            Come ogni altra cosa in ambito blockchain, una volta entrato a far parte dei quest'ultima
            il codice non può più essere modificato. \\
            Tuttavia gli smart contract hanno la possibilità di possedere uno stato interno che può essere
            alterato con l'invocazione di uno dei suoi metodi da parte di un \gls{eoa}.
        }
}
\newglossaryentry{eoa}
{
    name={Externally Owned Account (EOA)},
    first={Externally Owned Account (EOA)},
    text={EOA},
    plural={EOAs},
    description={
            Sono account della blockchain \gls{ethereum} identificati da un indirizzo pubblico associato ad una chiave privata.
            Chiunque sia in possesso della chiave privata è in grado di firmare transazioni a nome dell'account. \\
            Per invocare un metodo di uno \gls{smart-contract}, è necessario che la transazione sia stata originata da un account EOA.
        }
}
\newglossaryentry{dapp}
{
    name={Decentralized Application (DApp)},
    first={Decentralized Application (DApp)},
    text={DApp},
    plural={DApps},
    description={
            Le DApp si differenziano dalle applicazioni tradizionali in quanto scelgono di evitare il modello client-server
            per appoggiarsi su piattaforme blockchain e il loro network distribuito. \\
            Nella maggior parte dei casi si tratta di applicazioni web.
        }
}

